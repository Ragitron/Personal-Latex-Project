%!TEX program = xelatex
\documentclass[9pt, compress]{beamer}
\usetheme[titleprogressbar]{m}

\usepackage{booktabs}  
\usepackage[scale=2]{ccicons}
\usepackage{minted}
\usepgfplotslibrary{dateplot}
\usemintedstyle{trac}
\author{\textbf{Igar Tarasyuk}} 
\title{A Tale of Two Cases: Intracranial Pathology In The ED}
%\subtitle{}
%\logo{}
\institute{\textbf{The Townsville Hospital}}
\date{August 2018}
%\subject{}
%\setbeamercovered{transparent}
%\setbeamertemplate{navigation symbols}{}
\begin{document}
	\maketitle
	
    \begin{frame}
    	\frametitle{Case One}
        	Delayed presentation of a 73 year old male tourist with minimal medical history, experiencing less than four and half hours of left sided lower limb weakness resulting in a fall and resolving spontaneously
     \end{frame}
     \begin{frame}
     	History of Presenting Complaint
     	\begin{itemize}
        	\item 2/7 ago was bush walking listing to left side
            \item fell onto left side striking shoulder
            \item no headstrike, no other symptoms
            \item was able to mobilise shortly after, at baseline on presentation
            \item otherwise well
        \end{itemize}
     \end{frame}
     
     \begin{frame}
     	Background Medical History
        \begin{itemize}
        	\item Benign Prostatic Hypertrophy
            \item Palpitations with Anxiety
            \item TURP few years ago
            \item Medications: Paracetamol SR, and Propranolol PRN
            \item non-smoker, no family cardiac history, no previous CVA/TIA
        \end{itemize}
     \end{frame}
     
     \begin{frame}
     	Observations and Examination
        \begin{itemize}
        	\item Bright and Chatty, Alert and Orientated
            \item Vitals WNL, HR 71 Regular, BP 125/60
            \item unremarkable physical exam, no carotid bruits
            \item CNS and PNS examination normal
            \item sluggish and hesitant finger-to-nose in left
        \end{itemize}
     \end{frame}
     
     \section{What differentials present themselves in this case?}
     
     \begin{frame}
     	Investigations
        \begin{itemize}
        	\item CTB: No acute haemorrhage seen, Chronic microangiopathic change
            \item Bloods unremarkable
            \item ECG - normal sinus rhythm
        \end{itemize}
     \end{frame}
     
     \begin{frame}
     	\frametitle{Transient Ischaemic Attack}
     \end{frame}
     
     \begin{frame}
     	Follow-Up
        \begin{itemize}
        	\item Risk-Benefit of investigating in Townsville vs Home
            \item Opted for Carotid Doppler and TIA clinic
            \item Normal Carotids
            \item Commenced on Aspirin and Atorvastatin
            \item Return home with GP follow-up
            \item No driving for 2 weeks
        \end{itemize}
     \end{frame}
     
     \section{Thoughts?}
     
     \begin{frame}
     	\frametitle{Case Two}
        A 64 year old female visitor with multiple cardiac risk factors presenting with family concerns of personality and memory change, and one week of recurrent falls.
     \end{frame}
     
     \begin{frame}
     	History of Presenting Complaint
        \begin{itemize}
        	\item 1 month of notable personality change
            \item Increasingly forgetful over same time period
            \item 4 textit{intrinsic} falls over last week, misjudged distance to target
            \item Declining energy over last month and increasing SOB
            \item Recent LRTI managed with antibiotics
        \end{itemize}
     \end{frame}
     
     \begin{frame}
     	Background Medical History
        \begin{itemize}
        	\item Hypertension; Rx Irbesartan
            \item Hyperlipidaemia; Rx Rosuvastatin
            \item GORD; Rx Esomeprazole
            \item Chronic Back Pain; Rx Paracetamol/Codeine, Meloxicam
            \item Depression; Rx Mirtazapine, Oxazepam
            \item COPD; Rx Tiotropium
            \item Post-Menopausal; Rx Oestradiol patch
            \item Ex-Smoker, normally independent in all ADLs and assists at home
        \end{itemize}
     \end{frame}
     
     \begin{frame}
     	Observations and Examination
        	\begin{itemize}
            	\item Alert and orientated, easily interactive
                \item Unable to perform serial sevens
                \item Vitals WNL, BP 142/98, HR 75 regular
                \item Non-contributory physical exam
                \item L side mouth droop, otherwise normal neurology
                \item normal visual fields
            \end{itemize}
      \end{frame}
      
      \section{What work up does this patient require considering differentials?}
      
      \begin{frame}
      	Investigations
        \begin{itemize}
        	\item Bloods unremarkable
            \item ECG - normal sinus
            \item Urine M/C/S - no findings
            \item CXR: Right hilar lesion
            \item CTB: 3 intraparenchymal lesions with marked vasogenic oedema and mass effect, midline shift to left, subfalcine herniation, uncal herniation
        \end{itemize}
      \end{frame}
      
      \begin{frame}
      	\frametitle{Intracranial Metastases}
      \end{frame}
      
      \begin{frame}
      	Inpatient Investigation
        \begin{itemize}
        	\item CT CAP: Well defined 4.5*5*6.2 cm subpleural lesion encasing right lower bronchus, minor infiltration//
            6*11mm nodule in left upper lobe; 6*8mm nodule in right upper lobe//
            enlarged hilar lymph nodes//
            no abdominal or skeletal infiltrates identified
            \item MRI B: Multiple intraparenchymal metastases
            \item Bronchoscopy: 
     
     
\end{document}